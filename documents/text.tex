\documentclass[a4paper,article]{article}

% Шрифты
\usepackage{fontspec}
\usepackage[14pt]{extsizes}
\setmainfont{Times New Roman}

% Языки
% Русский обязательно идёт вторым. Иначе не работают переносы
\usepackage[english, russian]{babel}

% Параметры страницы
\usepackage[left=3cm, top=2cm, right=1.5cm, bottom=2cm]{geometry}
\usepackage[onehalfspacing]{setspace}

% Параметры текста
% По умолчанию LaTeX не делает отступ после \section. Вроде как оно и не надо, но в тексте ВКР пусть лучше будет. В требованиях отступ описан. Этот пакет своим наличием добавляет этот отступ
\usepackage{indentfirst}
% По умолчанию абзацный отстум меньше требуемого. Задаём конкретный
\setlength{\parindent}{1.25cm}

% Ссылки
\usepackage{color}
\definecolor{Black}{RGB}{0,0,0}
% Без colorlinks вокруг ссылок появляются рамки, недопустимые в ВКР
% Если не зачернить ссылки, то в оглавлении будет некрасивый красный цвет
\usepackage[colorlinks, linkcolor=Black]{hyperref}

% Пункты оглавления
\usepackage{titlesec}
\titleformat{\section}
{\centering\normalfont\bfseries}{\thesection. }{0em}{}
\titleformat{\subsection}
{\centering\normalfont\bfseries}{\thesubsection. }{0em}{}

% Таблицы
\usepackage{multicol}
\usepackage{xltabular}
\begin{document}
    \begin{titlepage}
        \begin{center}
            {\bfseries Министерство науки и высшего образования Российской Федерации \\
            Федеральное государственное автономное образовательное учреждение \\
            высшего образования \\
            <<КАЗАНСКИЙ (ПРИВОЛЖСКИЙ) ФЕДЕРАЛЬНЫЙ УНИВЕРСИТЕТ>>}
        \end{center}

        \begin{center}
            ИНСТИТУТ ВЫЧИСЛИТЕЛЬНОЙ МАТЕМАТИКИ И ИНФОРМАЦИОННЫХ ТЕХНОЛОГИЙ
        \end{center}

        \begin{center}
            КАФЕДРА АНАЛИЗА ДАННЫХ И ТЕХНОЛОГИЙ ПРОГРАММИРОВАНИЯ
        \end{center}

        \begin{center}
            Направление: 09.03.03 – Прикладная информатика
        \end{center}

        \vspace{0mm}

        \begin{center}
            ВЫПУСКНАЯ КВАЛИФИКАЦИОННАЯ РАБОТА \\
            {\bfseries СИСТЕМА ЗАПИСИ НА ПРИЁМ В МЕДИЦИНСКОЕ УЧРЕЖДЕНИЕ}
        \end{center}

        \vfill

        \begin{xltabular}{\textwidth} {
                >{\hsize=0.5\hsize} X
                >{\hsize=0.5\hsize} X }
            \bfseries{Работа завершена:} & \\
            Студент 4 курса & \\
            группы 09-951 & \\
            <<\underline{\hspace{1cm}}>>\underline{\hspace{3cm}} 2023г. & \underline{\hspace{3cm}}/Колесников Д.А. \\
            & \\
            \bfseries{Работа допущена к защите:} & \\
            Научный руководитель & \\
            старший преподаватель & \\
            <<\underline{\hspace{1cm}}>>\underline{\hspace{3cm}} 2023г. & \underline{\hspace{3cm}}/Еникеев И.А. \\
            & \\
            \multicolumn{2}{l}{Заведующий кафедрой анализа данных} \\
            и технологий программирования & \\
            <<\underline{\hspace{1cm}}>>\underline{\hspace{3cm}} 2023г. & \underline{\hspace{3cm}}/Бандеров В.В. \\
        \end{xltabular}

        \vspace{0mm}

        \begin{center}
            Казань — 2023
        \end{center}
    \end{titlepage}

    \pagebreak

    \setcounter{page}{2}

    \tableofcontents

    \pagebreak

    \section*{Введение}
    \addcontentsline{toc}{section}{Введение}

    \pagebreak

    \section{Предметная область}

    \subsection{Основные сведения}

    \subsection{Существующие решения}

    \subsection{Техническое задание}

    \pagestyle{plain}

    \pagebreak

    \section{Проектирование}

    \subsection{База данных}

    \subsection{Серверная часть}

    \subsection{Клиентская часть}

    \pagebreak

    \section{Реализация}

    \subsection{Выбор технологий}

    \subsection{База данных}

    \subsection{Серверная часть}

    \subsection{Клиентская часть}

    \pagebreak

    \section{Выпуск}

    \pagebreak

    \section*{Заключение}
    \addcontentsline{toc}{section}{Заключение}

    \pagebreak

    \section*{Список использованных источников}
    \addcontentsline{toc}{section}{Список использованных источников}

\end{document}
